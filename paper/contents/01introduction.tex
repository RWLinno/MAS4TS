\section{Introduction}

\begin{figure}[t!]
    \centering    
    \includegraphics[width=0.48\textwidth]{figures/comparison_methods.pdf}
    \caption{The motivation of \method framework. Existing methods (left: time-series specific methods; middle: pre-trained LM-based methods) face critical challenges while \method (right) leverages specialized agents to resolve them cooperatively through visual-semantic anchoring and numerical reasoning.}
    \label{fig:motivation}
    \vspace{-1em}
\end{figure}

% 第一段:时序分析的重要性
Time series analysis plays a crucial role in diverse real-world applications, including financial market prediction~\citep{idrees2019prediction}, medical diagnosis~\citep{karevan2020transductive}, energy management~\citep{deb2017review}, and IoT systems~\citep{zheng2020traffic}. The ability to accurately forecast, classify, impute, and detect anomalies in temporal data is essential for informed decision-making across these domains.

% 第二段:介绍LLM的发展和Agentic方法的兴起
Recent advances in pre-trained large language models (LLMs) have demonstrated remarkable capabilities in understanding and reasoning about complex patterns~\citep{brown2020language,touvron2023llama,dubois2023alpacafarm}. This success has naturally extended to time series analysis, where LLM-based methods~\citep{jin2023time,liu2024unitime,zhou2023one} have shown improved generalization and contextual understanding compared to traditional time-series specific models~\citep{wu2023timesnet,zhou2021informer,nie2022time}. Building upon this foundation, \emph{agentic approaches}~\citep{xi2023rise,wang2024survey} have emerged as a promising paradigm that leverages LLMs' tool-using capabilities and prior knowledge to autonomously solve tasks through structured workflows. More recently, \emph{multi-agent systems (MAS)}~\citep{wu2023autogen,hong2023metagpt,qian2024chatdev} have demonstrated that teams of specialized agents can collaboratively tackle complex problems that exceed the capabilities of any single agent, through task decomposition, parallel execution, and synergistic cooperation.

% 第三段:Challenges(重点改写)
However, applying these advances to time series analysis faces two fundamental challenges that limit practical deployment (Figure~\ref{fig:motivation}):

\vspace{-0.5em}
\begin{itemize}[leftmargin=*, itemsep=0pt]
    \item \textbf{Challenge 1: Weak Numerical Reasoning.} Existing methods—whether time-series specific models or LM-based approaches—primarily rely on \emph{over-parameterization} of historical data to capture patterns, lacking explicit numerical reasoning capabilities. Time-series specific models~\citep{wu2023timesnet,liu2023itransformer} learn patterns through extensive training on domain data but struggle to generalize across different domains and tasks without retraining. LM-based methods~\citep{jin2023time,zhou2023one} attempt to leverage pre-trained knowledge but face the \emph{modality gap}—converting continuous numerical time series into discrete tokens inevitably loses fine-grained temporal information, while pre-trained word embeddings are poorly aligned with numerical dynamics. Both approaches lack dedicated mechanisms for \emph{explicit numerical reasoning} that can verify predictions against statistical constraints, validate trend directions, or quantify uncertainty. 
    
    \item \textbf{Challenge 2: Inefficient Semantic-Temporal Alignment.} While semantic information (e.g., ``increasing trend'', ``seasonal pattern'') could guide time series analysis, current methods struggle with efficient semantic extraction and alignment. Text-based approaches~\citep{jin2023time,liu2024unitime} require costly text encoding of entire time series through LLMs, leading to slow inference and high computational overhead that hinders real-world deployment. Moreover, the \emph{semantic-temporal alignment} problem persists: textual descriptions of time series patterns (e.g., ``stock prices are rising'') often fail to capture precise numerical dynamics and timing, resulting in fragile connections between semantic understanding and actual temporal behavior.
\end{itemize}
\vspace{-0.5em}

% 第四段:我们的解决方案
To address these challenges, we introduce the first \underline{M}ulti-\underline{A}gent \underline{S}ystem framework for general \underline{T}ime \underline{S}eries analysis (\textbf{\method}), leveraging agentic collaboration to enable complex time series tasks to benefit from \emph{task decomposition} and \emph{specialized reasoning mechanisms}. Our key insight is that decomposing analysis workflows into coordinated subtasks—visual pattern recognition, numerical reasoning, and task-specific execution—allows each component to excel at its specialized role while collaborating synergistically. Specifically, as illustrated in Figure~\ref{fig:motivation}, \method comprises four core agents orchestrated by a central Manager: \textbf{(1)} the \emph{Data Analyzer Agent} extracts statistical features and conducts covariance-based top-$k$ feature selection to reduce computational complexity; \textbf{(2)} the \emph{Visual Anchor Agent} converts time series into images and generates prediction anchors (confidence intervals, trend directions, key points) through VLM-powered visual reasoning or efficient rule-based analysis, addressing Challenge 2 by enabling direct semantic-temporal alignment without costly text encoding; \textbf{(3)} the \emph{Numerologic Adapter Agent} performs explicit numerical reasoning by fusing visual anchors, statistical features, and semantic priors through attention-based multimodal fusion, optionally enhanced by LLM ensemble, thereby addressing Challenge 1 through dedicated numerical validation beyond pattern memorization; \textbf{(4)} the \emph{Task Executor Agent} invokes specialized time series models (e.g., DLinear, TimesNet, PatchTST) and applies anchor-based constraints to produce final predictions. The key advantage is \method's \emph{flexible efficiency-accuracy trade-off}: agents operate in lightweight mode (rule-based analysis) for \textbf{6.8$\times$} faster inference or full mode (VLM+LLM ensemble) for maximum accuracy, achieving \textbf{2.1$\times$} speedup over Time-LLM while delivering \textbf{12.3\%} better accuracy on average.

Our contributions are summarized as follows:

\vspace{-0.5em}
\begin{itemize}[leftmargin=*, itemsep=0pt]
    \item We propose MAS4TS, the first multi-agent system for general time series analysis, achieving competitive performance across four fundamental tasks: forecasting, classification, imputation, and anomaly detection.
    
    \item We introduce \emph{visual anchoring} and \emph{numerical reasoning} as key mechanisms: visual anchors provide semantic-temporal priors through efficient VLM-based (or rule-based) image analysis, while numerical reasoning explicitly validates predictions against multimodal constraints.

    \item We demonstrate superior efficiency-accuracy trade-offs compared to pre-trained LM-based methods, achieving \textbf{2.1$\times$ faster} inference and \textbf{62\% less} GPU memory while improving accuracy by \textbf{12.3\%} on average.

    \item Comprehensive experiments on 19 datasets across 4 tasks show MAS4TS establishes a novel paradigm for agentic time series analysis, with strong performance in few-shot and zero-shot scenarios.
\end{itemize}
\vspace{-0.5em}
